\section{User Manual}

If superuser, the head of command prompt is \verb|#|.
Otherwise, \verb|$|.


\subsection{API}


\subsection{Cofigurations}


Input \verb|make configure| and then \verb|make|.

\noindent
\begin{Sbox}
\begin{minipage}[t]{0.975\linewidth}
\begin{verbatim}
$ make configure
$ make
\end{verbatim}
\end{minipage}
\end{Sbox}
\fbox{\TheSbox}




\subsection{Setup}

Do program user level code in \verb|user| directory.


\subsection{How to Execute}

\noindent
\begin{Sbox}
\begin{minipage}[t]{0.975\linewidth}
\begin{verbatim}
# apt-get install ccache gcc clang
\end{verbatim}
\end{minipage}
\end{Sbox}
\fbox{\TheSbox}


\subsubsection{SIM}

\noindent
\begin{Sbox}
\begin{minipage}[t]{0.975\linewidth}
\begin{verbatim}
$ mcube [option]
\end{verbatim}
\end{minipage}
\end{Sbox}
\fbox{\TheSbox}


\subsubsection{ARM}


\noindent
\begin{Sbox}
\begin{minipage}[t]{0.975\linewidth}
\begin{verbatim}
# apt-get install libx11-dev libxpm-dev
# apt-get install gcc-aarch64-linux-gnu
\end{verbatim}
\end{minipage}
\end{Sbox}
\fbox{\TheSbox}




\subsubsection{X86}

\noindent
\begin{Sbox}
\begin{minipage}[t]{0.975\linewidth}
\begin{verbatim}
# apt-get install libc6-dev-i386 gcc-multilib
\end{verbatim}
\end{minipage}
\end{Sbox}
\fbox{\TheSbox}


\subsubsection*{QEMU}


If you would like to use QEMU 2.12 for Raspberry Pi3, please type as follows.

\noindent
\begin{Sbox}
\begin{minipage}[t]{0.975\linewidth}
\begin{verbatim}
# sudo apt build-dep qemu
# sudo apt install gcc-aarch64-linux-gnu
# CFLAGS=''-Wno-format-truncation'' ./configure --audio-drv-list=alsa
# make
# make install
\end{verbatim}
\end{minipage}
\end{Sbox}
\fbox{\TheSbox}

Otherwise, use apt-get.

\noindent
\begin{Sbox}
\begin{minipage}[t]{0.975\linewidth}
\begin{verbatim}
# apt-get install qemu qemu-system
\end{verbatim}
\end{minipage}
\end{Sbox}
\fbox{\TheSbox}


If you use VRAM, install VNC.

\noindent
\begin{Sbox}
\begin{minipage}[t]{0.975\linewidth}
\begin{verbatim}
# apt-get install gvncviewer vnc4server
\end{verbatim}
\end{minipage}
\end{Sbox}
\fbox{\TheSbox}

If you use other architectures except X86, download QEMU from
\url{http://wiki.qemu.org/Download}.
Next compile and install QEMU.


Execute QEMU emulator.

\noindent
\begin{Sbox}
\begin{minipage}[t]{0.975\linewidth}
\begin{verbatim}
$ qemu-system-i386 build/mcube.bin [-smp 1,cores=4]
\end{verbatim}
\end{minipage}
\end{Sbox}
\fbox{\TheSbox}

Display Monitor in VNC viewer.

\noindent
\begin{Sbox}
\begin{minipage}[t]{0.975\linewidth}
\begin{verbatim}
$ vncviewer 127.0.0.1:5900
\end{verbatim}
\end{minipage}
\end{Sbox}
\fbox{\TheSbox}


If you use serial, execute Mcube with below option.

\noindent
\begin{Sbox}
\begin{minipage}[t]{0.975\linewidth}
\begin{verbatim}
$ qemu build/mcube.bin -serial stdio
\end{verbatim}
\end{minipage}
\end{Sbox}
\fbox{\TheSbox}



\subsubsection*{VMWare}


\textbf{Guest Machine (Linux)}

First install samba.

\noindent
\begin{Sbox}
\begin{minipage}[t]{0.975\linewidth}
\begin{verbatim}
# apt-get install samba
\end{verbatim}
\end{minipage}
\end{Sbox}
\fbox{\TheSbox}

Next add samba user.

\noindent
\begin{Sbox}
\begin{minipage}[t]{0.975\linewidth}
\begin{verbatim}
# smbpasswd -a username
New SMB password:
Retype new SMB password:
Added user username
\end{verbatim}
\end{minipage}
\end{Sbox}
\fbox{\TheSbox}


\textbf{USB Boot on VMWare}


\noindent
\begin{Sbox}
\begin{minipage}[t]{0.975\linewidth}
\begin{verbatim}
# dd if=/home/chishiro/mcube/build/mcube-flat.vmdk of=/dev/sdb1
\end{verbatim}
\end{minipage}
\end{Sbox}
\fbox{\TheSbox}



\textbf{Host Machine (Windows)}

First install VMware player/workstation.
Next access guest machine's file system and open [mcube/vmdk/mcube.vmx].
At this time, you can execute Mcube!



\subsection{Precaution Statement}

\begin{enumerate}
 \setlength{\parskip}{0cm}
 \setlength{\itemsep}{0cm}
 \item Stack size is 0x400 byte and user should not use large size of
   local variables.
   User should use global variables.
   If stack size wants to be changed, edit
   \verb|#define STACK_SIZE 0x400| \verb|include/mcube/task.h|.
\end{enumerate}

\subsection{Emacs}

In order to compile Emacs, you must install tex package.

\noindent
\begin{Sbox}
\begin{minipage}[t]{0.975\linewidth}
\begin{verbatim}
# apt-get install libncurses5 libncurses5-dev autoconf
# apt-get install libx11-dev libxpm-dev libjpeg-dev libgif-dev
# apt-get install libgtk2.0-dev libglib2.0-dev libtiff4-dev
\end{verbatim}
\end{minipage}
\end{Sbox}
\fbox{\TheSbox}
% sudo apt-get install "libXpm*" "libpng*" "libgif*" "libungif*" "iibtfff*"

change from "deb http://xxx" to "deb-src http://xxx" in "/etc/apt/source.list".


\subsection{LaTeX}

In order to compile LaTeX, you must install tex package.

\noindent
\begin{Sbox}
\begin{minipage}[t]{0.975\linewidth}
\begin{verbatim}
# apt-get install tetex-*
# apt-get install texlive-latex* texlive-science
# apt-get install ptex-jtex okumura-clsfiles
# apt-get install xdvik-ja
# apt-get install texlive-fonts-recommended
# apt-get install vfdata-morisawa5 dvi2ps-fontdesc-morisawa5
\end{verbatim}
\end{minipage}
\end{Sbox}
\fbox{\TheSbox}


\subsection{Linux Kernel}

\noindent
\begin{Sbox}
\begin{minipage}[t]{0.975\linewidth}
\begin{verbatim}
# apt-get install linux-source-X.Y.Z
\end{verbatim}
\end{minipage}
\end{Sbox}
\fbox{\TheSbox}

\subsection{Multi-boot Linux}

\noindent
\begin{Sbox}
\begin{minipage}[t]{0.975\linewidth}
\begin{verbatim}
# emacs /etc/default/grub
\end{verbatim}
\end{minipage}
\end{Sbox}
\fbox{\TheSbox}

Edit \verb|GRUB_HIDDEN_TIMEOUT=10| and uncomment
\verb|GRUB_TERMINAL=console|.



\noindent
\begin{Sbox}
\begin{minipage}[t]{0.975\linewidth}
\begin{verbatim}
# update-grub
\end{verbatim}
\end{minipage}
\end{Sbox}
\fbox{\TheSbox}


\subsection{Doxygen}

\noindent
\begin{Sbox}
\begin{minipage}[t]{0.975\linewidth}
\begin{verbatim}
# apt-get install doxygen
\end{verbatim}
\end{minipage}
\end{Sbox}
\fbox{\TheSbox}



\subsection{Change Japanese to English Directory Names}

\noindent
\begin{Sbox}
\begin{minipage}[t]{0.975\linewidth}
\begin{verbatim}
$ LANG=C xdg-user-dirs-gtk-update
\end{verbatim}
\end{minipage}
\end{Sbox}
\fbox{\TheSbox}


\subsection{GNOME}

\noindent
\begin{Sbox}
\begin{minipage}[t]{0.975\linewidth}
\begin{verbatim}
# apt-get install aptitude
# aptitude -y install gnome 
\end{verbatim}
\end{minipage}
\end{Sbox}
\fbox{\TheSbox}


\subsection{Docker}

Docker is used without "sudo".

\noindent
\begin{Sbox}
\begin{minipage}[t]{0.975\linewidth}
\begin{verbatim}
# gpasswd -a $(whoami) docker
# chgrp docker /var/run/docker.sock
# service docker restart
\end{verbatim}
\end{minipage}
\end{Sbox}
\fbox{\TheSbox}

